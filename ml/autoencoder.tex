\documentclass[12pt]{article}
\usepackage{amsmath,amssymb,amsthm}
\usepackage{graphicx}
\usepackage{hyperref}
\usepackage{listings}
\usepackage{xcolor}

\newtheorem{theorem}{Theorem}
\newtheorem{definition}{Definition}

\title{71-Layer Autoencoder for LMFDB Compression}
\author{Meta-Introspector Research}
\date{January 29, 2026}

\begin{document}
\maketitle

\begin{abstract}
Follow-up experiment exploring neural network compression of LMFDB data using 
Monster prime dimensions. A 71-layer autoencoder achieves 23× compression with 
MSE of 0.233.

\textbf{Status}: Proof of concept, preliminary results.
\end{abstract}

\section{Architecture}

\subsection{The 71-Layer Autoencoder}

\begin{verbatim}
Input (5 dims)
    ↓
Encoder: 5 → 11 → 23 → 47 → 71
    ↓
Latent Space (71 dims)
    ↓
Decoder: 71 → 47 → 23 → 11 → 5
    ↓
Output (5 dims)
\end{verbatim}

All layer dimensions use Monster primes: $\{11, 23, 47, 71\}$

\subsection{Input Features}

Each LMFDB object encoded as 5-dimensional vector:

\begin{lstlisting}[language=Python]
class MonsterFeatures:
    number: float       # Normalized by 71
    j_invariant: float  # j(n) = (n^3 - 1728) mod 71
    module_rank: float  # Normalized by 10
    complexity: float   # Normalized by 100
    shard: float        # Shard ID mod 71
\end{lstlisting}

\section{Implementation}

\subsection{Python}

\begin{lstlisting}[language=Python]
class MonsterAutoencoder(nn.Module):
    def __init__(self):
        super().__init__()
        self.encoder = nn.Sequential(
            nn.Linear(5, 11), nn.ReLU(),
            nn.Linear(11, 23), nn.ReLU(),
            nn.Linear(23, 47), nn.ReLU(),
            nn.Linear(47, 71)
        )
        self.decoder = nn.Sequential(
            nn.Linear(71, 47), nn.ReLU(),
            nn.Linear(47, 23), nn.ReLU(),
            nn.Linear(23, 11), nn.ReLU(),
            nn.Linear(11, 5)
        )
\end{lstlisting}

\subsection{Rust}

\begin{lstlisting}[language=Rust]
struct MonsterAutoencoder {
    encoder_weights: Vec<Vec<Vec<f32>>>,
    decoder_weights: Vec<Vec<Vec<f32>>>,
}

impl MonsterAutoencoder {
    fn encode(&self, input: &[f32; 5]) -> Vec<f32> {
        let mut x = input.to_vec();
        for layer in &self.encoder_weights {
            x = self.apply_layer(&x, layer);
            x = self.relu(&x);
        }
        x
    }
}
\end{lstlisting}

\section{Results}

\subsection{Compression}

\begin{itemize}
\item Original data: 907,740 bytes (Parquet shards)
\item Compressed: 38,760 bytes (9,690 parameters × 4 bytes)
\item Ratio: 23.4×
\end{itemize}

\subsection{Reconstruction}

\begin{itemize}
\item MSE: 0.233
\item Training: 7,115 LMFDB objects
\item Latent dimensions: 71
\end{itemize}

\section{Future Work}

\begin{itemize}
\item Train on full LMFDB dataset
\item Compare to standard architectures
\item Ablation study: why Monster primes?
\item Statistical significance testing
\end{itemize}

\section{Code}

\begin{itemize}
\item Python: \texttt{monster\_autoencoder.py}
\item Rust: \texttt{lmfdb-rust/src/bin/monster\_autoencoder\_rust.rs}
\end{itemize}

\end{document}
