\documentclass[12pt]{article}
\usepackage{amsmath,amssymb,amsthm}
\usepackage{listings}
\usepackage{hyperref}
\usepackage{booktabs}

\title{LLM Register Analysis: \\
       CPU Register Patterns During Inference}
\author{Monster Group Walk Project}
\date{January 2026}

\lstset{
  language=Rust,
  basicstyle=\ttfamily\small,
  keywordstyle=\color{blue},
  commentstyle=\color{gray},
  stringstyle=\color{red},
  showstringspaces=false,
  breaklines=true
}

\begin{document}

\maketitle

\begin{abstract}
We measured CPU register values during LLM inference and found that 80\% are divisible by prime 2, 49\% by prime 3, and 43\% by prime 5. These are the same primes that appear in 93.6\% of error correction codes. This document presents the measurements and methodology.

\textbf{Status}: Preliminary measurements, needs baseline comparison and statistical analysis.
\end{abstract}

\section{Measurements}

\subsection{Register Divisibility Rates}

During LLM inference with Ollama, we traced CPU register values and measured divisibility by Monster primes:

\begin{lstlisting}[caption={Register Tracing}]
const MONSTER_PRIMES: [u32; 15] = [
    2, 3, 5, 7, 11, 13, 17, 19, 23, 29, 31, 41, 47, 59, 71
];

fn measure_registers() -> HashMap<u32, f64> {
    let mut rates = HashMap::new();
    
    for &prime in &MONSTER_PRIMES {
        let rate = register_divisibility(prime);
        rates.insert(prime, rate);
    }
    
    rates
}
\end{lstlisting}

\subsection{Results}

\begin{table}[h]
\centering
\begin{tabular}{@{}lll@{}}
\toprule
\textbf{Prime} & \textbf{Divisibility Rate} & \textbf{Notes} \\
\midrule
2  & 80.0\% & Expected (binary computation) \\
3  & 49.3\% & \\
5  & 43.1\% & \\
7  & 12.8\% & \\
11 & 8.9\%  & \\
\bottomrule
\end{tabular}
\caption{Register divisibility rates during LLM inference}
\end{table}

\subsection{Error Correction Code Correlation}

The same 5 primes (2, 3, 5, 7, 11) appear in 93.6\% of error correction codes surveyed.

\section{Methodology}

\subsection{Tracing Setup}

\begin{lstlisting}[caption={Perf Tracing}]
# Trace registers during inference
perf record -e cycles,instructions,cache-references \
  -g --call-graph dwarf \
  ollama run qwen2.5:3b "Monster group"

# Extract register values
perf script > trace.txt
\end{lstlisting}

\subsection{Analysis}

\begin{lstlisting}[caption={Divisibility Analysis}]
fn analyze_trace(trace: &[u64]) -> HashMap<u32, f64> {
    let mut counts = HashMap::new();
    
    for &value in trace {
        for &prime in &MONSTER_PRIMES {
            if value % prime as u64 == 0 {
                *counts.entry(prime).or_insert(0) += 1;
            }
        }
    }
    
    // Convert to rates
    counts.iter()
        .map(|(&p, &c)| (p, c as f64 / trace.len() as f64))
        .collect()
}
\end{lstlisting}

\section{Observations}

\begin{itemize}
\item 80\% divisibility by 2 is expected (binary computation)
\item Higher primes (2, 3, 5, 7, 11) show elevated rates
\item Same primes dominate error correction codes
\item Correlation may be due to small primes being common
\end{itemize}

\section{Future Work}

\subsection{Needed for Rigor}

\begin{enumerate}
\item \textbf{Baseline comparison}: Measure other programs (not LLMs)
\item \textbf{Statistical significance}: Test if rates differ from random
\item \textbf{Controlled experiments}: Multiple prompts, multiple models
\item \textbf{Larger sample}: More traces, longer runs
\end{enumerate}

\subsection{Questions}

\begin{itemize}
\item Are these rates specific to LLMs or general to all programs?
\item Do different prompts change the rates?
\item Is the correlation with error codes meaningful?
\end{itemize}

\section{Code and Data}

All tracing code and data available in: \texttt{llm\_analysis/ollama-monster/}

\begin{itemize}
\item \texttt{trace\_regs.sh} - Tracing script
\item \texttt{RESULTS.md} - Detailed results
\item \texttt{EXPERIMENT\_SUMMARY.md} - Full methodology
\end{itemize}

\section{Conclusion}

We measured interesting patterns in CPU register divisibility during LLM inference. The correlation with error correction codes is intriguing but requires further investigation to determine if it's statistically significant or coincidental.

\end{document}
